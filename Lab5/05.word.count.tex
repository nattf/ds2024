% 05.word.count.tex

\documentclass{article}
\usepackage{graphicx}

\title{MapReduce Word Count Report}
\author{Nguyen Ngoc Nhat Linh}
\date{24/04/2024}

\begin{document}

\maketitle

\section{Mapper and Reducer in Word Count}

In the Word Count example implemented using a MapReduce framework, both the Mapper and Reducer play crucial roles in the data processing pipeline.

\subsection{Mapper}

The Mapper component in the Word Count example takes the input data and processes it to produce intermediate key-value pairs. The key represents a word, and the value represents the count of that word. The Mapper performs the following steps:

\begin{enumerate}
  \item Read a line of text from the input data.
  \item Tokenize the line into individual words.
  \item For each word, emit a key-value pair where the key is the word and the value is 1.
\end{enumerate}

Figure \ref{fig:mapper} illustrates the working of the Mapper component in the Word Count example.

\begin{figure}[h]
  \centering
  \includegraphics[width=0.6\textwidth]{mapper.png}
  \caption{Mapper Working in Word Count}
  \label{fig:mapper}
\end{figure}

\subsection{Reducer}

The Reducer component in the Word Count example receives the intermediate key-value pairs produced by the Mapper and performs a reduction operation to aggregate values with the same key. The Reducer performs the following steps:

\begin{enumerate}
  \item Receive a key-value pair where the key is a word and the values are the counts for that word.
  \item Sum up the counts for each word.
  \item Emit a key-value pair where the key is the word and the value is the final word count.
\end{enumerate}

Figure \ref{fig:reducer} depicts the working of the Reducer component in the Word Count example.

\begin{figure}[h]
  \centering
  \includegraphics[width=0.6\textwidth]{reducer.png}
  \caption{Reducer Working in Word Count}
  \label{fig:reducer}
\end{figure}

\section{Responsibilities}

In the Word Count example implemented using a MapReduce framework, the responsibilities are divided as follows:

\begin{itemize}
  \item Mapper:
  \begin{itemize}
    \item Reads a line of text from the input data.
    \item Tokenizes the line into individual words.
    \item Emits a key-value pair for each word, where the key is the word and the value is 1.
  \end{itemize}
  \item Reducer:
  \begin{itemize}
    \item Receives intermediate key-value pairs from the Mapper.
    \item Aggregates the values for each key (word) by summing them up.
    \item Emits a key-value pair for each word, where the key is the word and the value is the final word count.
  \end{itemize}
\end{itemize}

The Mapper and Reducer components work together in the Word Count example to perform the necessary data processing and produce the final word count.

\end{document}